%%%%%%%%%%%%%%%%%%%%%%%%%%%%%%%%%%%%%%%%%%%%%%%%%%%%%%%%%%%%%%%%%%%%%%%%
%%%%%%%%%%%%%%%%%%%%%% Simple LaTeX CV Template %%%%%%%%%%%%%%%%%%%%%%%%
%%%%%%%%%%%%%%%%%%%%%%%%%%%%%%%%%%%%%%%%%%%%%%%%%%%%%%%%%%%%%%%%%%%%%%%%

%%%%%%%%%%%%%%%%%%%%%%%%%%%%%%%%%%%%%%%%%%%%%%%%%%%%%%%%%%%%%%%%%%%%%%%%
%% NOTE: If you find that it says                                     %%
%%                                                                    %%
%%                           1 of ??                                  %%
%%                                                                    %%
%% at the bottom of your first page, this means that the AUX file     %%
%% was not available when you ran LaTeX on this source. Simply RERUN  %%
%% LaTeX to get the ``??'' replaced with the number of the last page  %%
%% of the document. The AUX file will be generated on the first run   %%
%% of LaTeX and used on the second run to fill in all of the          %%
%% references.                                                        %%
%%%%%%%%%%%%%%%%%%%%%%%%%%%%%%%%%%%%%%%%%%%%%%%%%%%%%%%%%%%%%%%%%%%%%%%%

%%%%%%%%%%%%%%%%%%%%%%%%%%%% Document Setup %%%%%%%%%%%%%%%%%%%%%%%%%%%%

% Don't like 10pt? Try 11pt or 12pt
\documentclass[10pt]{article}

% This is a helpful package that puts math inside length specifications
\usepackage{calc}
% Layout: Puts the section titles on left side of page
\reversemarginpar

%
%         PAPER SIZE, PAGE NUMBER, AND DOCUMENT LAYOUT NOTES:
%
% The next \usepackage line changes the layout for CV style section
% headings as marginal notes. It also sets up the paper size as either
% letter or A4. By default, letter was used. If A4 paper is desired,
% comment out the letterpaper lines and uncomment the a4paper lines.
%
% As you can see, the margin widths and section title widths can be
% easily adjusted.
%
% ALSO: Notice that the includefoot option can be commented OUT in order
% to put the PAGE NUMBER *IN* the bottom margin. This will make the
% effective text area larger.
%
% IF YOU WISH TO REMOVE THE ``of LASTPAGE'' next to each page number,
% see the note about the +LP and -LP lines below. Comment out the +LP
% and uncomment the -LP.
%
% IF YOU WISH TO REMOVE PAGE NUMBERS, be sure that the includefoot line
% is uncommented and ALSO uncomment the \pagestyle{empty} a few lines
% below.
%

%% Use these lines for letter-sized paper
\usepackage[paper=letterpaper,
            includefoot, % Uncomment to put page number above margin
            marginparwidth=1.2in,     % Length of section titles
            marginparsep=.05in,       % Space between titles and text
            margin=1in,               % 1 inch margins
            includemp]{geometry}

%% Use these lines for A4-sized paper
%\usepackage[paper=a4paper,
%            %includefoot, % Uncomment to put page number above margin
%            marginparwidth=30.5mm,    % Length of section titles
%            marginparsep=1.5mm,       % Space between titles and text
%            margin=25mm,              % 25mm margins
%            includemp]{geometry}

%% More layout: Get rid of indenting throughout entire document
\setlength{\parindent}{0 in}

%% This gives us fun enumeration environments. compactitem will be nice.
\usepackage{paralist}

%% Reference the last page in the page number
%
% NOTE: comment the +LP line and uncomment the -LP line to have page
%       numbers without the ``of ##'' last page reference)
%
% NOTE: uncomment the \pagestyle{empty} line to get rid of all page
%       numbers (make sure includefoot is commented out above)
%
\usepackage{fancyhdr,lastpage}
\pagestyle{fancy}
\pagestyle{empty}      % Uncomment this to get rid of page numbers
\fancyhf{}\renewcommand{\headrulewidth}{0pt}
\fancyfootoffset{\marginparsep+\marginparwidth}
\newlength{\footpageshift}
\setlength{\footpageshift}
          {0.5\textwidth+0.5\marginparsep+0.5\marginparwidth-2in}
\lfoot{\hspace{\footpageshift}%
       \parbox{2in}{\, \hfill %
                    \arabic{page} of \protect\pageref*{LastPage} % +LP
%                    \arabic{page}                               % -LP
                    \hfill \,}}

% Finally, give us PDF bookmarks
\usepackage{color,hyperref}
\definecolor{darkblue}{rgb}{0.0,0.0,0.3}
\hypersetup{colorlinks,breaklinks,
            linkcolor=darkblue,urlcolor=darkblue,
            anchorcolor=darkblue,citecolor=darkblue}

%%%%%%%%%%%%%%%%%%%%%%%% End Document Setup %%%%%%%%%%%%%%%%%%%%%%%%%%%%


%%%%%%%%%%%%%%%%%%%%%%%%%%% Helper Commands %%%%%%%%%%%%%%%%%%%%%%%%%%%%

% The title (name) with a horizontal rule under it
%
% Usage: \makeheading{name}
%
% Place at top of document. It should be the first thing.
\newcommand{\makeheading}[1]%
        {\hspace*{-\marginparsep minus \marginparwidth}%
         \begin{minipage}[t]{\textwidth+\marginparwidth+\marginparsep}%
                {\large \bfseries #1}\\[-0.15\baselineskip]%
                 \rule{\columnwidth}{1pt}%
         \end{minipage}}

% The section headings
%
% Usage: \section{section name}
%
% Follow this section IMMEDIATELY with the first line of the section
% text. Do not put whitespace in between. That is, do this:
%
%       \section{My Information}
%       Here is my information.
%
% and NOT this:
%
%       \section{My Information}
%
%       Here is my information.
%
% Otherwise the top of the section header will not line up with the top
% of the section. Of course, using a single comment character (%) on
% empty lines allows for the function of the first example with the
% readability of the second example.
\renewcommand{\section}[2]%
        {\pagebreak[2]\vspace{1.3\baselineskip}%
         \phantomsection\addcontentsline{toc}{section}{#1}%
         \hspace{0in}%
         \marginpar{
         \raggedright \scshape #1}#2}

% An itemize-style list with lots of space between items
\newenvironment{outerlist}[1][\enskip\textbullet]%
        {\begin{itemize}[#1]}{\end{itemize}%
         \vspace{-.6\baselineskip}}

% An environment IDENTICAL to outerlist that has better pre-list spacing
% when used as the first thing in a \section
\newenvironment{lonelist}[1][\enskip\textbullet]%
        {\vspace{-\baselineskip}\begin{list}{#1}{%
        \setlength{\partopsep}{0pt}%
        \setlength{\topsep}{0pt}}}
        {\end{list}\vspace{-.6\baselineskip}}

% An itemize-style list with little space between items
\newenvironment{innerlist}[1][\enskip\textbullet]%
        {\begin{compactitem}[#1]}{\end{compactitem}}

\catcode240=13 \def �{\u g}
\catcode231=13 \def �{\c c}
\catcode246=13 \def �{\"o}
\catcode254=13 \def �{\c s}
\catcode252=13 \def �{\"u}
\catcode253=13 \def �{{\i}}
\catcode221=13 \def �{\.I}
\catcode199=13 \def �{\c C}
\catcode208=13 \def �{\u G}
\catcode214=13 \def �{\"O}
\catcode222=13 \def �{\c S}
\catcode220=13 \def �{\"U}

% To add some paragraph space between lines.
% This also tells LaTeX to preferably break a page on one of these gaps
% if there is a needed pagebreak nearby.
\newcommand{\blankline}{\quad\pagebreak[2]}

%%%%%%%%%%%%%%%%%%%%%%%% End Helper Commands %%%%%%%%%%%%%%%%%%%%%%%%%%%

%%%%%%%%%%%%%%%%%%%%%%%%% Begin CV Document %%%%%%%%%%%%%%%%%%%%%%%%%%%%

\begin{document}
% NOTE: Mind where the & separators and \\ breaks are in the following
\begin{center}{\textbf{BLG372E Analysis of Algorithms}}\\
\vspace{5mm}
\normalsize{Homework 2}} \\
%\normalsize{website: \href{http://www.onurvarol.com}{http://www.onurvarol.com}} \\
\vspace{3mm}
\normalsize{{Ozan Arkan CAN}\\ 
\normalsize{040090573}
\end{center}
\hrulefill
%       table.
%
% ALSO: \rcollength is the width of the right column of the table
%       (adjust it to your liking; default is 1.85in).
%
\newlength{\rcollength}\setlength{\rcollength}{1.5in}%
%
%\begin{tabular}[t] {@{}p{\textwidth-\rcollength}p{\rcollength}}

\section{PROBLEM 1} 
%
In genereal, order of functions like that:\\

$log(n) \leq n^{x} \leq r^{n} \leq n^{n}$ , $\forall x > 0$ and  $r > 1$ \\

We can order given functions by this order information.\\\\
\begin{math}
F_9 = 2/n < F_{12} = 37 < F_2 = \sqrt{n} < F_1 = n < F_6 = nlog(logn) < F_5 = nlogn < F_8 = nlog(n^2) < F_7 = nlog^2n < F_3 = n^{1.5} < F_4 = n^2 < F_{13} = n^2logn < F_{14} = n^3 < F_{11} = 2^{n/2} < F_{10} = 2^n\\   
\end{math}

\\$F_9$ is a descending function, so its complexity minimum. $F_{12}$ is a constant function and it never grows. Its obvious that relation between $\sqrt{n}$ , $n$ , $nlog(logn)$ and $nlogn$. If we apart function $F_8$ and $F_7$ like $n$ times $log(n^2)$ , $log^2n$; those part of functions act logaritmic and pow of number. $F_3$, $F_4$ , $F_{13}$ and $F_{14}$ functions are ordered by their power and $log(n) < n$ information. $F_{11}$ and $F_{10}$ functions are ordered by their power.  

\section{PROBLEM 2}
First of all, given algorithm does not work all possible situation. For n = 16 and maximum independent set size = 11, k gets following values:\\\\
8, 12, 6, 9, 13, 7, 10, 15, 8\\\\
There is infinite loop and exceeding loop counter ( $log_2n$ ).\\

Algorithm checks there is an independent set of size k in a graph in the loop. But k is not constant, so for this line complexity is not $O(n^k)$. k can be $O(n)$ . Thus, complexity goes $O(n^n)$ . I assume algorithm terminates $O(log_2n)$ by ignoring bug of algorithm. In this situation, complexty is $O(log_2n*n^n)$ . $(n^2)*2^n < log_2n*n^n$ . Given algorithm is slower than the slides in worst case.

\section{PROBLEM 3}
If we investigate each iteration of outer loop, we can see the second loop after outer loop iterates $n - 1$ . 
If we look inner loop iteration by iteration, total number of iteration is 1 for $S_1$ , 2 for $S_2$ , 3 for $S_3$ ,..., n for $S_n$ . Here, $S_i$ belongs to outer loop. Thus, total number of iteration of whole algorithm is; \\
\\$(n-1)*1 + (n-1)*2 + ... + (n-1)*n$\\
$
(n-1)* \frac{n*(n+1)}{2}
$\\

So complexity is;\\

$ 
(n-1)* \frac{n*(n+1)}{2} = O(n^3)
$\\
\section{PROBLEM 4}
\begin{outerlist}
\item a) It is true.\\
$c_1*h \leq f \leq c_2*h$ $n \geq n_0
\\ d_1*h \leq g \leq d_2*h$ $n \geq n_1
\\ c_1*d_1*h*h \leq f*g \leq c_2*d_2*h*h$ , $n \geq max(n_0, n_1)$
\\ $f*g = \theta(h*h)$
\\ $c_1*d_1$ and $c_2*d_2$ are constant.

\item b)\\
$log_2(n) \leq c*n^{0.5}$ , $n \geq n_0$\\
$n \leq 2^{c*n^{0.5}}$ , $n \geq n_0$\\
$c = 1$ , $n_0 = 0 \Rightarrow n \leq 2^{c*n^{0.5}}$ , $\forall n$ $n \geq n_0$

\end{outerlist}
%\section{Conducted Researches and Academic Experience}
%\textbf{Research}
%\begin{outerlist}
%\item Research on Protein dynamics and mode-coupling analysis
%\item Research on Transcriptional regulatory networks.
%\item Research on Sketch Recognition using Dynamic Programming.
%\item Research on Modelling of Social Networks and Phase Transitions of Complex Systems
%\begin{innerlist}	
%        				\item Advisor:
%              			\href{http://web.itu.edu.tr/~erzan/}
%                   			{Prof.  Ay{s}e Erzan}        
%        				\end{innerlist} 
%\item Research on EEG Signal Processing and Classification.
%\begin{innerlist}	
%        				\item Advisor:
%              			\href{http://web.itu.edu.tr/~yalcinmust/}
%                   			{Assoc. Prof. M��tak Erhan Yal\c{c}{\i}n}        
%        				\end{innerlist} 
\end{outerlist}
\end{document}
%%%%%%%%%%%%%%%%%%%%%%%%%% End CV Document %%%%%%%%%%%%%%%%%%%%%%%%%%%%%